
\begin{document}
\documentclass[letterpaper, 12pt, oneside]{article}%para dar formato al documento

\usepackage{amsmath}
\usepackage{graphicx}
\usepackage{xcolor}
\usepackage[utf8]{inputenc}
\usepackage{enumitem}

%Aquí inicia la portada de mi documentos
\title{\Huge Taller de Herramientas Computacionaes}
\author{Jorge S. Martínez Villafan}
\date{Enero 24, 2019}

\begin{document}
En el problema 8 de python facil, solo cree una lista, copié el codigo que viene en el libro y lo corrí. Lo que hace el codigo es recorrer la lista dos veces una con indice i y otra con j. A estos terminos se le saca la diferencia de i con j y se le da el valor absoluto. Un nombre que le daría sería, "Valor absoluto de la diferencia"
\end{document}
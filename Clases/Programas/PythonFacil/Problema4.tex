\documentclass[letterpaper, 12pt, oneside]{article}%para dar formato al documento

\usepackage{amsmath}
\usepackage{graphicx}
\usepackage{xcolor}
\usepackage[utf8]{inputenc}
\usepackage{enumitem}

%Aquí inicia la portada de mi documentos
\title{\Huge Taller de Herramientas Computacionaes}
\author{Jorge S. Martínez Villafan}
\date{Enero 24, 2019}

\begin{document}
\documentclass[letterpaper, 12pt, oneside]{article}%para dar formato al documento

\usepackage{amsmath}
\usepackage{graphicx}
\usepackage{xcolor}
\usepackage[utf8]{inputenc}
\usepackage{enumitem}

%Aquí inicia la portada de mi documentos
\title{\Huge Taller de Herramientas Computacionaes}
\author{Jorge S. Martínez Villafan}
\date{Enero 21, 2019}

\begin{document}
En el problema cuatro de python facil defino una funcion y doy una variable, hago lo que esta en el libro y doy condiciones si el reciduo de i y x es cero, entonces son multiplios\\
 Nuevamente hago lo del libro y pido que se vaya acumulando la suma 
\end{document}
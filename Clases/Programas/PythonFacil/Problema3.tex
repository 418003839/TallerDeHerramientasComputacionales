\documentclass[letterpaper, 12pt, oneside]{article}%para dar formato al documento

\usepackage{amsmath}
\usepackage{graphicx}
\usepackage{xcolor}
\usepackage[utf8]{inputenc}
\usepackage{enumitem}

%Aquí inicia la portada de mi documentos
\title{\Huge Taller de Herramientas Computacionaes}
\author{Jorge S. Martínez Villafan}
\date{Enero 24, 2019}

\begin{document}
\documentclass[letterpaper, 12pt, oneside]{article}%para dar formato al documento

\usepackage{amsmath}
\usepackage{graphicx}
\usepackage{xcolor}
\usepackage[utf8]{inputenc}
\usepackage{enumitem}

%Aquí inicia la portada de mi documentos
\title{\Huge Taller de Herramientas Computacionaes}
\author{Jorge S. Martínez Villafan}
\date{Enero 21, 2019}

\begin{document}
En el problema 3 de python facil primero defino una variable
con un input pido el numero a evaluar y con un  for hago que pase por 1 y n+1 posteriormente comparo su reciduo. Si es 0 entonces es dibisible y doy una asignacion para que verifique cuandos divisores tiene\\
La asignacion debe ser distinto de 2. porque si es 2, entonces se divide el solo y 1. Si tiene más de dos imprimo ("No es primo") pero si tiene solo dos divisores imprimo ("Es primo") 
\end{document}
\end{document}
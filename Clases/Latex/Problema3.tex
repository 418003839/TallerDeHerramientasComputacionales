\documentclass[letterpaper, 12pt, oneside]{article}%para dar formato al documento
\usepackage{amsmath}
\usepackage{graphicx}
\usepackage{xcolor}
\usepackage[utf8]{inputenc}
\usepackage{enumitem}

%Aquí inicia la portada de mi documentos
\title{\Huge Taller de Herramientas Computacionaes}
\author{Jorge S. Martínez Villafan}
\date{Enero 21, 2019}

\begin{document}
\maketitle
%includegraphics[scale=0.4]{1.png}
\newpage
\title{Resumen del problema 3}
\section{Tarea4}
\textbf En el problema 3 con ayuda de un input pedía los grados que se desearan convertir de Celsius a Farenheit y viceversa, se definió  una función para F y C, y un print para cada uno


\section{Tarea5}
Creé una lista vacía y con mediante el input le pedí al usuario que me diera valores enteros, a esta asignación le llame C\\
Definí a F como la función con la cual se pasan de F a C, al resultado de esto le asigné la letra \color{red} "A" \\
\color{black} Analogamente hice lo mismo con la función "C" para pasar de C a F, a este resultado le asigné la letra \color{red} "B" \color{black} pedí que se imprimera la lista [A,B]
\end{document}

\documentclass[letterpaper, 12pt, oneside]{article}%para dar formato al documento
\usepackage{amsmath}
\usepackage{graphicx}
\usepackage{xcolor}
\usepackage[utf8]{inputenc}
\usepackage{enumitem}

%Aquí inicia la portada de mi documentos
\title{\Huge Taller de Herramientas Computacionaes}
\author{Jorge S. Martínez Villafan}
\date{Enero 24, 2019}

\begin{document}
\maketitle
\newpage
\title{Clase número catorce}

\textbf Empezamos la clase numero catorce resolviendo dudas de la tarea de los ejercicios de la clase\\
Le pedimos ayuda al profesor para resolver el problema del laberinto  y explicó nuevamente cómo debía de hacerse, y con los pasos que siempre se utilizaban en el curso, desde analizar, ejemplificar, codificar etc.\\
De la pizarra pasamos a la computadora y nos fue guiando como es que debe de resolverse:\\
El codigo fue el siguiente: \\
\begin{verbatim}

def resolver(L,e):
print e
n = len(L[0])
m = len(L)
x = e[0]
y = e[1]
if y==n-1 or x == m-1: 
return e[0]+1,e[1]+1
else:
if L[x][y+1] == False:
e = [x, y+1]
return resolver(L,e)
elif L[x+1][y] == False:
e=[x+1,y]
return resolver(L,e)
else:
print ("ya no puede avanzar mas")

L= [[True, True, True, True],
[False, False, False,True],
[True, True, False, True]]  
e=[1,0]
r=resolver(L,e)
import numpy as np
print(np.matrix(L))
\end{verbatim}
Con algunas condiciones más dentro del laberinto el profesor nos pidió que trataramos de generar el codigo para poder resolver laberintos más complejos\\

Empezamos a trabajar con una problema sobre las cadenas de ADN y generamos el siguiente código 
\begin{verbatim}

def contar_v1(adn, base):
adn=list(adn)
i= 0
for c in adn:
if c == base:
i += 1
return i

def contar_v2(adn, base):
adn=list(adn)
i= 0
for c in adn:
if c == base:
i += 1
return i

def contar_v3(adn, base):
i= 0
for j in range(len(adn)):
if adn[j] == base:
i += 1
return i

def contar_v4(adn, base):
adn=list(adn)
i = 0
j = 0 
while j < len(adn):
if adn[j] == base:
i += 1
j +=1
return i

adn="ATGCGACCTAT"
base="C"
print contar_v1(adn, base)
print contar_v2(adn, base)

n= contar_v2(adn, base)
print n
print "%s aparece %d en %s" % (base, n, adn)
print "{base} aparece {n} veces en {adn}" .format(base=base, n=n, adn=adn)
print contar_v2(adn, base)
print contar_v3(adn, base)
print contar_v4(adn, base)
\end{verbatim}
Este codigo nos sirvió para que nos explicara lo que hace el comando \color{blue} sum \color{black} que sobre una lista bool solo nos devuelves los valores que son "True"\\

La penultima clase del curso finalizó.
%Utilizo \begin{verbatim} para poder escribir los codigos sin que haya problemas por los carácteres que se usan en TeXstudio
%Utilizo \\ para hacer un salto de línea
%Utilizo \color{color} para cambiar de color a las letras
%utilizo \begin{enumerate} para hacer una lista enumerada

\end{document}
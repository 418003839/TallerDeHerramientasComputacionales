\documentclass[letterpaper, 12pt, oneside]{article}%para dar formato al documento
\usepackage{amsmath}
\usepackage{graphicx}
\usepackage{xcolor}
\usepackage[utf8]{inputenc}
\usepackage{enumitem}

%Aquí inicia la portada de mi documentos
\title{\Huge Taller de Herramientas Computacionaes}
\author{Jorge S. Martínez Villafan}
\date{Enero 11, 2019}

\begin{document}
\maketitle
%includegraphics[scale=0.4]{1.png}
\newpage
\title{Cuestinario}

\begin{enumerate}
	\item Menciona algunos sistemas operativos\\
R: Windows, Linux, iOs
\item Menciona algunos lenguajes de programación\\
R: Python, C, C++, java
\item Menciona las distintas versiones de Linux\\
R: Ubuntu, Fedora,Debian y Slakw.
\item ¿Qué es un bit?\\
R: es un cero o un uno, debido al sistema binaro en el que trabajan las computadoras
\item Menciona los tres tipos de permisos que exiten en Linux\\
	Usuario (rwx), Grupo (rw-) y Todos (r--)
\item ¿Qué es un shell?\\
Un interprete de comandos
\item ¿Qué es un PATH?\\
Una variable de entorno
\item ¿Que hace el comando touch /tmp/algo?\\ 
Crea un archivo vacio
\item ¿Y el comando ls - l /tmp/algo?\\ 
Arroja información sobre permisos
\item ¿El comando chmod 751 /tmp/algo?\\
Cambia los permisos de usuarios, grupos y todos
\item ¿El comando /top qué hace?\\ 
Muestra información sobre la computadora como CPU y su actividad
\item ¿y el comando cd/?\\ 
Muestra el directorio de raíz
\item ¿Con que comando puedo ver el contenido de un directorio?\\
Con el comando ls
\item ¿Son importantes el uso de minusculas y mayusculas al escribir en la terminal de Linux?\\
Sí, se debe ser cuidadoso porque las mayúsculas y minúsculas son distintos simbolos para la terminal
\item ¿Qué es github?
Un servidor que permite bajar, subir, compartir e intercambiar información de la nube, bastante utilizado por los programadores.
\item ¿Cómo instalar github en Fedora?\\
git init\\
sudo apt-git upgrade\\
sudo apt-git install git\\
\item ¿Cómo subir un archivo a github?\\
 git add *\\ git commit\\ git push\\usuario\\contraseña
\item ¿Como clonamos nuestro repositorio en una maquina nueva?\\
Con el comando git clone y pegando el link de nuestro repositorio de github
\item ¿Cómo deben ser guardados los archivos generados en latex?\\
Con .tex
\item ¿Y los de python?\\
Con .py
\item ¿Con qué comando creamos una nueva carpeta?\\
Con mkdir - p
\item ¿Para qué sirve el comando wq?\\
Para guardar un comentario y salir
\item ¿Y el comando q!?\\
Para salir sin guardar
\item ¿Cómo instalar python en fedora?\\
Con el comando \$] dnf install pyhton tools
\item ¿Cómo sabemos con que versión de python trabajamos?\\
\$] python --visor\\
\item ¿Con que comando abrimos python desde la terminal?\\
Con idle
\item ¿Cómo hacemos que pyhton deje de arrojarnos divisiones enteras?\\
Basta con agregar un .0 al divisor o dividendo
\item ¿Cómo escribir un comentario en python?\\
Con un \# presediendo al comentario
\item ¿Qué debemos poner al inicio del archivo si queremos escribir con carácteres que no existen en el inglés?\\
\#\_*\_ conding: uft-8 \_*\_
\item ¿Qué es una cadena en python?\\
Un conjunto de carácteres que van delimitados por una comilla, comilla doble o triple
\item ¿Qué son los flotantes?\\
Una representación de los numeros reales en Python
\item  ¿Para qué siver print en Python?\\ 
Sirve para mostrar lo que pedimos
\item ¿Para qué siver el \%?\\ 
Sirve para desplegar una variable a través de print. Se combina con varias letras (E,f,g), para desplegar un valor en distintos formatos
\item ¿Y el comando \%g?\\ 
Para mostrar la variable en el formato numerico más corto posible
\item ¿el comando\%E qué hace?\\ 
Muestra el formato en notación cientifica
\item ¿Y el comando \% .f?\\
 Muestra un valor con flotante con dos de su decimales (puede ser cualquier valor decimal)
\item ¿Qué pasa con el comando \%10.f?\\ 
Muestra un numero flotante con dos decimales recorrido 10 espacios a la derecha
\item ¿Cómo desplegamos el valor en formato flotante?\\
Con \%f
\item ¿Qué hacemos con \%s?\\
Si tengo una variable que contenga una cadena y quiero mostrar su contenido con print se utiliza este comando
\item ¿Qué comando usamos para sacar raíces cuadradas?\\
 \$] math.sqrt
\item  ¿Cómo definimos una función?\\
con el comando def
\item ¿Cómo idicamos que regrese algún valor? \\
Con return  
\item ¿Cómo iniciamos un documento nuevo en TeXstudiio? \\ Con documentclass
\item ¿Cómo ponemos la letra que deseamos?\\
Basta con poner [letterpaper,numero,lugar]
\item ¿Cómo importamos algo en TeXstudio?\\
Con usepackage{amsmath} 
\item ¿Para qué sirve usepackage{graphicx}\\
Para las graficas 
\item ¿Y usepackage{xcolor}?\\
Para poder utilizar letras de colores
\item ¿Qué debemos hacer para poder escribir carácteres que no existen en el inglés?\\
Poner usepackage[utf8]{inputenc}
\item ¿Para qué sirve usepackage{enumitem}\\ 
Para poder enumerar 
\item ¿Cómo añadimos un comentario?\\
Con \%
\end{enumerate}
\end{document}
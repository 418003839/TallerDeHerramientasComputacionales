
\documentclass[letterpaper, 12pt, oneside]{article}%para dar formato al documento
\usepackage{amsmath}
\usepackage{graphicx}
\usepackage{xcolor}
\usepackage[utf8]{inputenc}
\usepackage{enumitem}


%Aquí inicia la portada de mi documentos
\title{\Huge Taller de Herramientas Computacionaes}
\author{Jorge S. Martínez Villafan}
\date{Enero 14, 2019}

\begin{document}
\maketitle
%includegraphics[scale=0.4]{1.png}
\newpage
\title{Clase numero tres}

\textbf{Al inicio de la tercer clase del curso se empezó a hablar de los servidores y sus clientes, por ejemplo Faceebook o el mismo github}

Se hablaron sobre las distintas terminaciones de algunos archivos, por ejemplo los archivos generados en Latex deben ir terminados con un .tex y los archivos generados en Python con un .py

La clase prosiguió haciendo, basicamente, un repaso de lo visto en la clase numero dos. Hicimos ejercicios similares a los de la clase pásada. 
El profesor nos enseñó a hacer carpetas utilizando la terminas, a su vez, estas carpetas servirían para guardar los archivos que hicieramos durante el curso. Los pasos para hacer dichas carpetas son las siguientes:
\begin{enumerate}
	\item escribir en la termina mkdir - p directorio/ directorio esto crea un directorio para todas, es decir un directorio padre
	\item nombrar las carpetas que queremos crear 
	\item abrir las carpetas creadas con cd "nombre de la carpeta" por si queremos crear otra carpeta dentro de la abierta
	

\end{enumerate}

Estas carpetas fueron subidas al github, para saltarnos el paso de escribir el comentario, basta con poner git commit -m "comentario".
Para salir de la pantalla del comentario se presiona la tecla "esc", seguido por shit más dos puntos y finalizamos con "wq" si queremos guardar y salir o "q!" si queremos no guardar y salir.

Posteriormente vimos lo que era el "vi" que crea un documento y creamos un readme, con el comando "readme.md"


La clase finalizó con el profesor pidiendonos llevar un problema de fisica para la clase próxima.
\end{document}
%\documentclass{article}
\documentclass[letterpaper, 12pt, oneside]{article}%para dar formato al documento
\usepackage{amsmath}
\usepackage{graphicx}
\usepackage{xcolor}
\usepackage[utf8]{inputenc}
\usepackage{enumitem}


%Aquí inicia la portada de mi documentos
\title{\Huge Taller de Herramientas Computacionaes}
\author{Jorge S. Martínez Villafan}
\date{Enero 19, 2019}

\begin{document}
\maketitle
%includegraphics[scale=0.4]{1.png}
\newpage
\title{Cuestionario de la clase 12}

\textbf{Segundo cuestionario clase 12}

\begin{enumerate}
\item ¿El modulo pprint para qué sirve?\\
R: \color{red} para darle un formato distinto a las listas \color{black}
\item ¿Si metemos el siguiente comando qué hará A[2:]?\\
R: \color{red} muestra los valores apartir del indice dos \color{black}
\item ¿Si metemos el siguiente comando qué hará A[2:3]?\\
\color{red} nos regresa los valores a partir del indice dos pero hasta el 3 \color{black}
\item ¿Si metemos el siguiente comando qué hará A[:3]?\\
\color{red} devuelve los valores anteriores al indice 3 \color{black}
\item ¿Si metemos el siguiente comando qué hará A[1:-1]?\\
\color{red} regresará todos menos el primero y el ultimo \color{black}
\item ¿Cuál es la diferencia entre for in range y for in "lista"?\\
\color{red} el primero recorre la lista por indice, el segundo por sus valores \color{black}
\item ¿qué hace la coma al final?\\
\color{red} Omite el salto de línea \color{black}
\item ¿Cómo podemos poner un tema dentro de una diapositiva dentro de TeXstudio?\\
\color{red} Con el paquete usetheme \color{black}
\item ¿Comó hacemos una transición dentro de una diapostiva?\\ 
\color{red} con el comando trans \color{black}
\item ¿Para qué sirve frametitle?\\
\color{red} para titular la diapositiva previamente creada \color{black}
\item ¿Para qué sirve el paquete documentclass{beamer}?\\
\color{red} para poder trabajar con diapostivas 

\end{enumerate}
\end{document}
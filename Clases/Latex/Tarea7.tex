\documentclass[letterpaper, 12pt, oneside]{article}%para dar formato al documento
\usepackage{amsmath}
\usepackage{graphicx}
\usepackage{xcolor}
\usepackage[utf8]{inputenc}
\usepackage{enumitem}

%Aquí inicia la portada de mi documentos
\title{\Huge Taller de Herramientas Computacionaes}
\author{Jorge S. Martínez Villafan}
\date{Enero 24, 2019}

\begin{document}
\maketitle
%includegraphics[scale=0.4]{1.png}
\newpage
\title{Resumen de la tarea 7}
\section{Tarea7}
\textbf La tarea 7 constaba de hacer un ejercicio con las acciones de las listas que no vimos en clase.\\
\begin{enumerate}
	\item \color{red} .copy \color{black} sirve hacer una copia de la lista
	\item \color{red} .sort \color{black} sirve para poder ordenar la lista
	\item \color{red} .reverse \color{black} ordena la lista inversamente
	\item \color{red} .remove \color{black} remueve el elemento de x que le digas
\end{enumerate}
Con esto hice un ejercicio en el que daba una lista pedía que se hiciera una copia de ella, se ordenara numericamente, se ordenara inversamente, contara cuantos numeros dos había en la lista y posteriomente se eliminaran los 4 y 5
\end{document}
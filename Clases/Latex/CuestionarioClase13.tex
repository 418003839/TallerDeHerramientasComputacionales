\documentclass[letterpaper, 12pt, oneside]{article}%para dar formato al documento
\usepackage{amsmath}
\usepackage{graphicx}
\usepackage{xcolor}
\usepackage[utf8]{inputenc}
\usepackage{enumitem}


%Aquí inicia la portada de mi documentos
\title{\Huge Taller de Herramientas Computacionaes}
\author{Jorge S. Martínez Villafan}
\date{Enero 19, 2019}

\begin{document}
\begin{enumerate}
\item ¿Cuando se dice que una función en recursiva?\\
\color{red} Cuandos se define en terminos de sí misma \color{black}
\item ¿Cuales variables de entorno existen en python?\\ 
\color{red} Las variables globales y locales \color{black}
\item ¿Cuales son las variables locales? \\
\color{red} son las variables que pertencen a una función \color{black}
\item ¿Cuales son las variables globales? \\
\color{red} son las variables que responden a todas las funciones \color{black}
\item ¿Se puede visualizar como corre un codigo en python?\\ 
\color{red} Sí, con la siguiente pagina  {http://www.pythontutor.com/visualize.html} \color{black}
\item ¿Puede existir una lista infinita?\\
\color{red} en teoria no.\color{black}
\item ¿Por qué?\\
\color{red} Porque los recursos de la computadora son finitos

\end{enumerate}
\end{document}
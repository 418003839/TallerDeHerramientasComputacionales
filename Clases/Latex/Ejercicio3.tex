\documentclass{book}
\usepackage[spanish]{babel}
\usepackage[utf8]{inputenc}
\usepackage{biblatex}
%\usepackage{hyperref}

\title{ Taller de Herramientas Computacionaes}
\author{Jorge S. Martínez Villafan}
\date{Enero 17, 2019}

\begin{document}
\maketitle
%Aquí inicia el indice del contenido del texto.
\tableofcontents
\section*{Introdución} El libro es para fortalecer el conocimiento de la materia taller de herramientas computacionales.
\url{www.google.com}
\hyperref[Google]{www.google.com}
\chapter{Uso básico de Linux}
\section{Distribuciones de Linux}
\section{Comandos}

%Aquí inician los capitulos del lubro
\chapter{Introducción a LateX}
\chapter{Introdución a Python}
\begin{verbatim}


#!/usr/bin/python2.7
# -*- coding: utf-8 -*-
"""
Jorge S Martínez Villafan, 418003839
Herramientas computacionales
Lo que nos explico el miercoles
"""
x = 10.5;y = 1.0/3;z = 15.3
H = """
El punto de R3 es:
(x,y,z)=(%.2f,%g,%G)
""" % (x,y,z)
print H

G="""
El punto en R3 es:
(x,y,z)=({laX:.2f}.{laY:g},{laZ:G})
""".format(laX=x,laY=y,laZ=z)

print G

import math as m
from math import sqrt
from math import sqrt as s
from math import *
x=16
x=input("Cuál es el valor al que le quieres\n" + "calcular la raíz")
print "La raiz cuadrada de %.2f es %f" % (x,m.sqrt(x))
print sqrt(16.5)
print s(16.5)
\end{verbatim}

\input{/home/thc/Documentos/TallerDeHerramientasComputacionales/TallerDeHerramientasComputacionales/Clases/Latex/Prueeba.py}
\input{Prueeba.py}
\section{Orientacion a Objetos}

\begin{thebibliography}{9}
%\bibitem{Computación}
Autor yo mero alv\\
\textit{Cualquier cosa} 2019
\end{thebibliography}
	
	
\end{thebibliography}

\end{document}



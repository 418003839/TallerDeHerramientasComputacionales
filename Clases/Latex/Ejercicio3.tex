\documentclass{book}
\usepackage[spanish]{babel}
\usepackage[utf8]{inputenc}
%\usepackage{biblatex}
\usepackage{hyperref}

\title{ Taller de Herramientas Computacionaes}
\author{Jorge S. Martínez Villafan}
\date{Enero 17, 2019}

\begin{document}
\maketitle
%Aquí inicia el indice del contenido del texto.
\tableofcontents
\section*{Introdución} El libro es para fortalecer el conocimiento de la materia taller de herramientas computacionales.
\url{www.google.com}
\hyperref[Google]{www.google.com}
\chapter{Uso básico de Linux}
\section{Distribuciones de Linux}
\section{Comandos}

%Aquí inician los capitulos del lubro
\chapter{Introducción a LateX}
\chapter{Introdución a Python}
\begin{verbatim}


#!/usr/bin/python2.7
# -*- coding: utf-8 -*-
"""
Jorge S Martínez Villafan, 418003839
Herramientas computacionales
Lo que nos explico el miercoles
"""
x = 10.5;y = 1.0/3;z = 15.3
H = """
El punto de R3 es:
(x,y,z)=(%.2f,%g,%G)
""" % (x,y,z)
print H

G="""
El punto en R3 es:
(x,y,z)=({laX:.2f}.{laY:g},{laZ:G})
""".format(laX=x,laY=y,laZ=z)

print G

import math as m
from math import sqrt
from math import sqrt as s
from math import *
x=16
x=input("Cuál es el valor al que le quieres\n" + "calcular la raíz")
print "La raiz cuadrada de %.2f es %f" % (x,m.sqrt(x))
print sqrt(16.5)
print s(16.5)
\end{verbatim}

%\input{/home/jorgesmartinez/Documentos/TallerDeHerramientasComputacionales/Clases/Latex/Prueeba.py}
\documentclass[letterpaper, 12pt, oneside]{article}%para dar formato al documento
\usepackage{amsmath}
\usepackage{graphicx}
\usepackage{xcolor}
\usepackage[utf8]{inputenc}
\usepackage{enumitem}

%Aquí inicia la portada de mi documentos
\title{\Huge Taller de Herramientas Computacionaes}
\author{Jorge S. Martínez Villafan}
\date{Enero 21, 2019}

\begin{document}
\maketitle
%includegraphics[scale=0.4]{1.png}
\newpage
\title{Resumen del problema 1}
\section{Tarea4}
\textbf El problema numero uno pedía calcular el máximo común divisor mediante el método de euclides\\
Lo primero que hice fue definir una función en donde se ocuparan dos numeros, condicioné a que si el numero dos era igual a cero entonces nos devolviera el numero 1. De lo contrario se haria una comparación
Después se le da la idicación que nos devuelva la funcion aplicada al numero dos y el residuo del numero uno y dos

\section{Tarea5}
Di una lista con dos numeros, le dije que si el indice 1, era igual a 0 entonces nos devolviera el indice 0. sino, nos devuelva la función aplicada al indice 0 y el residuo del indice 0 y 1. 
\end{document}
\section{Orientacion a Objetos}

\begin{thebibliography}{9}
%\bibitem{Computación}
Autor yo mero alv\\
\textit{Cualquier cosa} 2019
\end{thebibliography}
	
	


\end{document}



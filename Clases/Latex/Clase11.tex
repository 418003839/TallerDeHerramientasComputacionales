%\documentclass{article}
\documentclass[letterpaper, 12pt, oneside]{article}%para dar formato al documento
\usepackage{amsmath}
\usepackage{graphicx}
\usepackage{xcolor}
\usepackage[utf8]{inputenc}
\usepackage{enumitem}

%Aquí inicia la portada de mi documentos
\title{\Huge Taller de Herramientas Computacionaes}
\author{Jorge S. Martínez Villafan}
\date{Enero 11, 2019}

\begin{document}
\maketitle
%includegraphics[scale=0.4]{1.png}
\newpage
\title{Clase número once}

\textbf En la clase numero once del curso iniciamos pregutandole dudas al profesor de los 10 problemas que debía resolverse con listas.
Posteriormente el profesor nos enseñó a sacar promedios utilizando las listas, teníamos que recorrer la lista con el metodo 	\color{blue}for i in "nombre" \color{black}, después dimos una asignación de a la letra r para que en ella se asignara lo que deseabamos. Y volvivimos a utilizar \color{blue} for in \color{black} para hacer que el valor fuera recorriendo la lista y la asignación r cabiaria su a ella misma más el valor.\\
Para que el resultado de la división nos diera un decimal, de ser el caso que así fuera utilizamos el comando \color{blue} Float, \color{black} dividido por \color{blue} len \color{black} que es la longitud de la lista.\\

Hicimos una archivo llamado \color{blue} gitignore \color{black}para poder evitar que se suabar archivos no deseados como:
\begin{enumerate}
\item .aux 
\item .pdf 
\item.syntec.gz 
\item .pyc
\end{enumerate}
Continuamos trabajando por python y nos enseñó el comando \color{blue} for in \color{black} que debemos usar una i porque generalemente usamos esta letra para contar.\\

Nos explicó cuales son las sumas superiores e inferiores para las integrales y nos pidió calcular la área de 2x+5x\^2-6x\^3+12
de [a,b].\\

\color{blue}range[(len)]\color{black}, significa que len está anidado a range y primero se ejecuta len, después range

No pidiio que de tarea checaramos para que sirven los demas L. y hacer un ejempolo con ellos

\color{blue}Enumerate \color{black} da el indice y el valor de la lista en ese indice\\
Los corchetes sirven para acceder a un elemento de un lista o definir una lista\\
Una \color{red}tupla \color{black} es como una lista donde no se pueden cambiar los elementos \\
Para finalizar nos pidió hacer una lista con varias litas dentro con los valores de farenheit en centigrados\\

%Utilizo \\ para hacer un salto de línea
%Utilizo \color{color} para cambiar de color a las letras
%utilizo \begin{enumerate} para hacer una lista enumerada

\end{document}
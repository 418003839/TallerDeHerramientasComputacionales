
\documentclass[letterpaper, 12pt, oneside]{article}%para dar formato al documento
\usepackage{amsmath}
\usepackage{graphicx}
\usepackage{xcolor}
\usepackage[utf8]{inputenc}
\usepackage{enumitem}


%Aquí inicia la portada de mi documentos
\title{\Huge Taller de Herramientas Computacionaes}
\author{Jorge S. Martínez Villafan}
\date{Enero 14, 2019}

\begin{document}
\maketitle
%includegraphics[scale=0.4]{1.png}
\newpage
\title{Segundo día de clases}

\textbf{En el segundo día del curso iniciamos probando los comando que vimos en el día anterior, pudimos saber la utilidad  de varios comandos como:}


\begin{enumerate}
	\item touch /tmp/algo crea un archivo vacio
	\item ls - l /tmp/algo arroja información sobre permisos
	\item chmod 751 /tmp/algo cambia los permisos de usuarios, grupos y todos
	\item /top muestra información sobre la computadora como CPU y su actividad
	\item cd/ muestra el directorio de raíz
	
La clase prosiguió con una explicación tanto del profesor como el de la ayudante Karla sobre qué era Git Hub:
Github es un servidor que permite bajar, subir, compartir e intercambiar información de la nube, bastante utilizado por los programadores.

Todos los alumnos nos registramos en la página de github utilizando nuestros correos de ciencias y nuestro numero de clave: 
utilizando los siguientes comandos en la terminal de Linux

	
	\begin{enumerate}
		\item git config --global user.email "ejemplo@ciencias.unam.mx"
		\item git config --global user.name "numero de cuenta"

		\end{enumerate}
Si queremos bajar nuestra información a una computadora nueva, lo único que debemos hacer, una sola vez, es poner el comando "git clone", pegar el link de nuestro respotirio de github y así se descargarán nuestros archivos a la computadora deseada

Algunos otros comandos del github que vimos durante la clase del martes pasado fuero
	
	\begin{enumerate}
		\item cd "nombre de la carpeta deseada" para abrir una carpeta
		\item ls para visualizar lo que hay en la carpeta anteriormente abierta
		\item git add * para agregar nuevos archivos al repositorio de github
		\item git commit para finalizar el agregado
		 
	\end{enumerate}
Para instalar github en fedora, se deben escribir lo siguiente en la terminal: 
\begin{enumerate}
	\item git init
	\item sudo apt-git upgrade
	\item sudo apt-git install git
\end{enumerate}


Para subir nuevos archivos a github
Se debe abrir las carpetas donde se tiene los archivos deseados. 
Escribir en la terminal:
\begin{enumerate}
	\item git add *
	\item git commit
	\item git push
	\item usuario
	\item contraseña
\end{enumerate}

\end{enumerate}
\end{document}
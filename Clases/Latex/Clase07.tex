\documentclass[letterpaper, 12pt, oneside]{article}%para dar formato al documento
\usepackage{amsmath}
\usepackage{graphicx}
\usepackage{xcolor}
\usepackage[utf8]{inputenc}
\usepackage{enumitem}

%Aquí inicia la portada de mi documentos
\title{\Huge Taller de Herramientas Computacionaes}
\author{Jorge S. Martínez Villafan}
\date{Enero 12, 2019}

\begin{document}
\maketitle
%includegraphics[scale=0.4]{1.png}
\newpage
\title{Clase número siete}

\textbf La clase numero siete del curso del Taller de Herramientas Computacionales fue el día martes.
Esta clase inició con ejemplos de como utilizar estructuras de repetición, en este caso se utlizó el comando "while", este comando se encarga de formar un ciclo, primero evalua si la condición dada fue cierta el ciclo se repetirá hasta que le digamos que pare. \\
Los comando if y else que vimos una clase pásada son muy utiles a la hora de utilizar while, se le puede dar una condición con if para que el ciclo se repita si el "if" fue cierto (aunque realmente también se podría si fue falso), y poner un else para, por ejemplo detener el ciclo. \\
Es muy importante utilizar los bloques y espacios correctamente en el uso del while. Las condiciones en un bloque while deben ir en las líneas inferiores y estar con cuatro espacios de distancias del inicio de documento. Tanto if como else deben ir dentro del bloque while si éstas son codicones que queremos que el código corra con while. If y else deben estar a la misma altura, pero ambas deben estar en distinta altura de While si pertencen a este comando. \\
El profesor hizo enfasís en que al momento de dar un asignación, (ejemplo: a=b+c), los espacios no son importantes pero se recomienda ponerlos para poder facilitar la lectura tanto del programador como la del lector\\

La clase prosiguió y fue turno de Karla quién nos enseñó varias cosas que podemos hacer en TeXstudio. \\
Iniciamos tratando de poner una imagen, pero lamentablemente no se pudo, debido a un problema con la dirección donde se encontraban la imagen que queríamos, aunque algunos pudieron soucionarlo con poner una tilde y cambiar de lugar la ubicación de la imagen\\
Dejando de lado lo de la imagen, nos enfocamos en cómo colocar expresiones matemáticas en TeXstudio:
\begin{enumerate}
\item \$ \$ sirven para indicar que escribiremos expresiones, matemáticas, existen las "diagonalCorchete" que son analogas al signo de moneda
\item \$x\_{2}\$ para poner subindices, en este caso, es equis subindice dos
\item \$x\^{2}\$ sirven para poner exponentes, en este caso equis al cuadrado
\item \$\\frac{2}{2}\$ sirven para poner fraciones, en este caso, dos partidos dos
\item \ begin{bmatrix} sirve para poner matrices, los terminos de éstas deben ir con un \& entre ellos para poder diferenciarlos
\item dots puntos suspensivos
\item vdots puntos suspensivos  verticales
\item sqrt para poder escribir raices (similar a pyhon)
\item \$\\int\_{a}\^{b}x\^2 dx, para poder escribir integrales, en este caso, la integral de a a b, de equis cuadrada


\end{enumerate}
La clase finalizó
\end{document}
\documentclass[letterpaper, 12pt, oneside]{article}%para dar formato al documento
\usepackage{amsmath}
\usepackage{graphicx}
\usepackage{xcolor}
\usepackage[utf8]{inputenc}
\usepackage{enumitem}


%Aquí inicia la portada de mi documentos
\title{\Huge Taller de Herramientas Computacionaes}
\author{Jorge S. Martínez Villafan}
\date{Enero 15, 2019}

\begin{document}
\maketitle
%includegraphics[scale=0.4]{1.png}
\newpage
\title{Clase numero cinco}

\textbf En la clase numero cinco vimos varios comandos de python, que básicamente la explicación de ellos podría resumir gran parte de la clase, los comando fueron los siguientes:
\begin{enumerate}
	\item \$] print que sirve para mostrar lo que pedimos, el profesor suele explicarlo, como "piensa en un numero y dime cuál es"
	\item \$]\% sirve para desplegar una variable a través de print. Se combina con varias letras (E,f,g), para desplegar un valor en distintos formatos
	\item \$]\%g mostrar la variable en el formato numerico más corto posible
	\item \$]\%E muestra el formato en notación cientifica
	\item \$]\% .f muestra un valor con flotante con dos de su decimales (puede ser cualquier valor decimal)
	\item \$]\%10.f muestra un numero flotante con dos decimales recorrido 10 espacios a la derecha
	\item \$]\%f desplega el valor en formato flotante
	\item \$]\%s si tengo una variable que contenga una cadena y quiero mostrar su contenido con print se utiliza este comando
	\item \$] math.sqrt para sacar raíces cuadradas
	\item \$] def para definir una función
	\item \$] return indica que regrese un valor
\end{enumerate}
Continuamos la clase con un archivo que copiamos de la pantalla del profesor con ayuda del tiger, bastaba con solo dar clic medio con el raton para poder pegar lo que el profesor subrayó. A dicho archivo tradujimos los comandos que venían en él.


La clase prosiguió y fue turno de Karla, quien nos enseñó los primeros pasos para iniciar en Latex
Nos pidió abrir TeXstudio
y nos empezó a decir los comandos y para qué sirve cada uno
\begin{enumerate}
    \item documentclass[letterpaper, 12pt, oneside] Para iniciar y darle el formato de texto
	\item usepackage{amsmath} analogo al import en python
	\item usepackage{graphicx} para las graficas 
	\item usepackage{xcolor} para poder utilizar letras de colores
	\item usepackage[utf8]{inputenc} para poder escribir carácteres que no existen en el inglés como la dieresis o las tikdes
	\item usepackage{enumitem} para poder enumerar 
	\item \% para poder añadir un comentario, es analogo al \# en python
\end{enumerate}
Y eso fue lo último que hicimos en la primer semana del curso
\end{document}
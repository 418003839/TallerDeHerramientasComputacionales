\documentclass[letterpaper, 12pt, oneside]{article}%para dar formato al documento
\usepackage{amsmath}
\usepackage{graphicx}
\usepackage{xcolor}
\usepackage[utf8]{inputenc}
\usepackage{enumitem}

%Aquí inicia la portada de mi documentos
\title{\Huge Taller de Herramientas Computacionaes}
\author{Jorge S. Martínez Villafan}
\date{Enero 21, 2019}

\begin{document}
\maketitle
%includegraphics[scale=0.4]{1.png}
\newpage
\title{Resumen del problema 2}
\section{Tarea4}
\textbf En el problema dos, como se tenían varias constante como la velocidad inicial y la gravedad solo bastaba con sustituir el tiempo Con input pedí dos valores al usario y estos se resolverían en la función y 

\section{Tarea5}
Di una lista con tres numeros, el velocidad inicial, la gravedad y un determinado tiempo. Definí a los indices del uno al dos como la velocidad inicial, la gravedad y el tiempo respectivamente. Definí la función con estos indices. El programa no copiló.
\end{document}
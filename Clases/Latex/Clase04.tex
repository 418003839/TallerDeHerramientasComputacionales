
\documentclass[letterpaper, 12pt, oneside]{article}%para dar formato al documento
\usepackage{amsmath}
\usepackage{graphicx}
\usepackage{xcolor}
\usepackage[utf8]{inputenc}
\usepackage{enumitem}


%Aquí inicia la portada de mi documentos
\title{\Huge Taller de Herramientas Computacionaes}
\author{Jorge S. Martínez Villafan}
\date{Enero 14, 2019}

\begin{document}
\maketitle
%includegraphics[scale=0.4]{1.png}
\newpage
\title{Clase numero cuatro}

\textbf La cuarta clase del curso la iniciamos trabajando con un problema en que el profesor llevo, similar al que nos había encargado la clase pásada, en él nos pedían determinar la posición de un objeto si sabemos la formula para calcularla. El profesor desglosó cada asunto que se pudiera presentar en el problema, desde si podía ser negativa, como su función. La analizó desde un punto matemático como desde un objeto para darle explicaciones a la computadora de lo que debe hacer. El profesor hizo un gran enfasis en esto ultimo ya que la computadora entiende exactamente lo que se le dice, es por ello que debemos ser claros y explicar detalladamente lo que queremos que haga

Las personas que no utilizaban las maquinas de la escuela tuvieron que instalar python. Fedora ya viene con python pero para poder abrirlo se deben escribir en la terminal el siguiente comando: \$] dnf install pyhton tools \\ 

Una vez con python todos, pudimos ver la versión de python escibiendo el comando \$] python --visor\\
Posteriormente abrimos pyton para meter la formula del problema del profesor, pyton se abre con el comando "idle" en la terminal. Cuando abrimos pyton prosedimos a meter los datos que teníamos, hubo un pequeño problema con el resultado debido a que pyton nos arrojaba divisiones enteras, para arreglar este problema solo se debe agredar un .0 después del divisor o dividendo. 
Durante la resolución del ejercicio el profesor nos dio varios tips acerca de como escribir en pyton por ejemplo para agregar un comentario éste debe ser presedido por el signo gato (\#), o para escribir en español correctamente, es decir, que pyton reconozca caracteres que en el inglés no existen, como la dieresis o las tildes, se debe agregar al inicio el siguiente código: \#\_*\_ conding: uft-8 \_*\_
También nos enseñó que para correr un códgio en pyton debe ir precedido por el comando print
La clase porsiguió con el profesor dandonos instruciones sobre como realizar una cadena en pyton  
\begin{enumerate}
	\item Comilla simple '' sirve para crear una cadena de texto de una línea
	\item Comilla doble "" sirve para crea una cadena de texto de una línea
	\item Comilla triple "' sirve para crea una cadena de texto multiple
	\item ln: para salto de línea
	
NOTA: una cadena es un conjunto de caracteres en pyton que están delimitados por una comilla o comilla doble
	
La clase finalizó realizando la tarea en el salón, con el problema que nos encargó en la clase numero tres debíamos generar un codigo para resolverlo de una manera similar al problema que el profesor resolvió durante la clase. Una vez hecho se subió como es acostumbrado a github.
\end{enumerate}
\end{document}
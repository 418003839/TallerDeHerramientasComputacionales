\documentclass[letterpaper, 12pt, oneside]{article}%para dar formato al documento
\usepackage{amsmath}
\usepackage{graphicx}
\usepackage{xcolor}
\usepackage[utf8]{inputenc}
\usepackage{enumitem}

%Aquí inicia la portada de mi documentos
\title{\Huge Taller de Herramientas Computacionaes}
\author{Jorge S. Martínez Villafan}
\date{Enero 14, 2019}

\begin{document}
\maketitle
%includegraphics[scale=0.4]{1.png}
\newpage
\title{Clase número seis}

\textbf La cuarta clase del curso inició con un probelma, trata de encontrar la raíz cuadrada de un, el problema radicaba en encontrar la raíz de equis, si tenemos dos  figuras geometricas de área "X", la primera era un cuadrado con los lados raíz de equis y la segunda un rectangulo de altura equis y base 1. Todo el grupo dio diversas sugerencias para poder encontrar el valor de equis, finalmente el profe dio el metodo para encontrar el valor de dicha incognita. Haciendo un gran enfasís que en las computadoras es dificíl encontrar el valor exacto, pero podemos encontrar una aproximación precisa que nos sirva (en realidad creo que esto sucede en las computadora y en los humanos ya que nunca sabemos el valor de un numero racional raíz de dos o pi, son algunos ejemplos en donde no se utilizan los valores exactos, solo aproximaciones).\\
La clase prosiguió con un problema en pyhon, donde definimos una función llamada "vAbsoluto", es decir, el valor absoluto, este ejercicio sirvió como ejemplo para poder utilizar dos comandos muy utiles en python el
\begin{enumerate}
\item if
\item else
\item return
\end{enumerate}
Estos comandos son codiciones:\\
El comando if, sirve para dar una sentencia, ejemplo: "Si sucede tal cosa" entonces (print) pasa esto\\
El comando else, es para arrojar otro resultado, si el comando if resulta ser falso\\
Tanto if como else tiene el mismo valor, es por ello que ambos deben ir a la misma altura del código y en el mismo bloque. Es importante resaltar que se puede utilizar un if, sin un else, pero es imposible colocar un else si un if, ya que else es una especie de negación del "if"\\
El comando return sirva para que nos devuelva el valor que hemos asigando previamente
También vimos el comando import que sirve para mostrar lo que deseemos en el bash\\

Después de haber explicado el uso de if y else el profesor nos puso un ejercicio, donde se tenía una diana infinita y al arrojar los dardos nos arrojaba un valor dependiendo del las condernas tanto en x como en y que hubiesemos sacado\\
La clase finalizó con el profesor dejandonos una tarea del mismo problema de la diana pero con más condiciones, si en el centro de la diana hubiera un circulo de radio 10 y fuera de éste el valor que nos arrojara fuera 100, pero estando delimitado por las otras coordenadas.
\end{document}
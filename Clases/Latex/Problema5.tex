\documentclass[letterpaper, 12pt, oneside]{article}%para dar formato al documento
\usepackage{amsmath}
\usepackage{graphicx}
\usepackage{xcolor}
\usepackage[utf8]{inputenc}
\usepackage{enumitem}

%Aquí inicia la portada de mi documentos
\title{\Huge Taller de Herramientas Computacionaes}
\author{Jorge S. Martínez Villafan}
\date{Enero 21, 2019}

\begin{document}
\maketitle
%includegraphics[scale=0.4]{1.png}
\newpage
\title{Resumen del problema 5}
\section{Tarea4}
\textbf En el problema 5 primero definí una función llamada sumar(n), y con ayuda de return le pedí que devuelva el termino de la operación de gauss, con un input pedí los numeros que se deseaban sumar y asigné a suma la suma desde n, utilice un print para que lanzara el resultado
\section{Tarea5}
Definí la función sumar y pedí que me devolviera el resultado de la formula gaussiana. Con \color{blue} input \color{black} pedí que el usuario me diera la cantidad de naturales que quiera sumar. Cree una lista vavía donde se alcenaria la suma que definí como sumar(n). El resultado se arroja en una lista.

\end{document}
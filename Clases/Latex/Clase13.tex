\documentclass[letterpaper, 12pt, oneside]{article}%para dar formato al documento
\usepackage{amsmath}
\usepackage{graphicx}
\usepackage{xcolor}
\usepackage[utf8]{inputenc}
\usepackage{enumitem}

%Aquí inicia la portada de mi documentos
\title{\Huge Taller de Herramientas Computacionaes}
\author{Jorge S. Martínez Villafan}
\date{Enero 23, 2019}

\begin{document}
\maketitle
\newpage
\title{Clase número trece}

\textbf Iniciamos la clase leyendo los problemas de la pagina 58 del libro python facil, el profesor nos pidió que leyeramos los probleas y ver si lo podriamos resolverlos o no.\\
Hubo problemas con el ejercicio de las matrices porque no lo hermos visto.\\ 
También el salon en general le pidió ayuda con el ejercicio del laberinto que tambíen utilizaba matrices. Con ayuda de todo el grupo el profesor nos explicó como se debe resolver con 4 pasos. Desde situarse en la entrada, pedirle que avance, hacia el frente, si puede, sino dar ordenes que explore por otros caminos como arriba, abajo, o atras, con preguntas como: ¿Ya llegué a la salida? y ¿puedo avanzar?.\\

Se disciutió acerca de si puede existir una lista infinita en una computadora, la mayoria estuvo de acuerdo de que no porque en la computadora los reales no existen, y los recursos de una computadora son finitos, pero se podría hacer un programa que arroje datos hasta que se le dé la indicación de parar con el comando \color{blue}ctrl + c \color{black}. En el libro python fácil venía un ejercicio en el cual se realiza una "lista infinita", la copiamos y tratamos de correr, algunos tuvieron problemas pero otros pocos pudieron correrla realizando algunas modificaciones.\\
Los 8 problemas del libro python facil quedaron de tarea\\
El profesor empezó a resolver el problema de fibonacci de manera recursiva, no sin antes enseñarnos lo qué es una formula recurisiva, es decir, una formula que está definida en terminos de ella misma.\\
También se resolvió la suma de n de manera recursiva \\
Le pregunté hacerca del juedo "Los postes de hanoi" y el profesor explicó en qué cosite el juego: Se tienen 3 postes un con discos de distitos tamañanos, ordenados desde el mayor a menor tamaño, utilizando uno de los postes vacios como auxiliar, se deben pasar los discos de un poste a otro, pero no deben quedar un disco mayor encima de un disco menor. \\
Dio una idea de cómo se puede resolver utilizando python, los postes seran listas, los discos seran numeros uno mayor que otro. Los discos de numero 0 equivaldra a que no hay discos en la torres de hanoi.\\
Nos pidipo hacer que una lista muetre el primer elemento primero y después el resto mediante una forma recursiva, muchos tuvimos problemas al generar el codigo, el profesor terminó dando la respuesta y al salón le quedó claro\\

Solo existen los variables en ambitos de validesm las variables locales y variables locales\\
Las variables locales cuando la variable pertenece a una funcion\\
Las variables globales todas las funciones responden a la variable\\
Continuamos la clase y el profesor nos mostró en el tiger una página que sirve para poder correr un codigo en python  y otros lenguajes de programación y poder ver en tiempo real cómo el programa se va ejecutando\\
La página es la siguiente: url http://www.pythontutor.com/visualize.html

La clase finalizó
%traté de utilizar \url o hyperrey para poder citar la pagina web pero creo que no tengo instalado esa biblioteca porque no me deja correrlo
%Utilizo \\ para hacer un salto de línea
%Utilizo \color{color} para cambiar de color a las letras
%utilizo \begin{enumerate} para hacer una lista enumerada

\end{document}
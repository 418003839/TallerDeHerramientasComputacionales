\documentclass[letterpaper, 12pt, oneside]{article}%para dar formato al documento
\usepackage{amsmath}
\usepackage{graphicx}
\usepackage{xcolor}
\usepackage[utf8]{inputenc}
\usepackage{enumitem}


%Aquí inicia la portada de mi documentos
\title{\Huge Taller de Herramientas Computacionaes}
\author{Jorge S. Martínez Villafan}
\date{Enero 14, 2019}

\begin{document}
\maketitle
%includegraphics[scale=0.4]{1.png}
\newpage
\title{En el primer día de clases, el profesor nos habló, sobre lo que se vería en el curso, tanto de las formas de trabajar como el metodo de calificación.}

\textbf{Lo que hice en esta primera clase del curso fue:}
\begin{enumerate}
	\item  El curso conta de lo siguiente:
	\begin{enumerate}
		\item Hacer un bitacora diariamente de lo visto en clase, exponer sus dudas
		\item Quince asistencias totales durante las 3 semanas que dura el curso
		\item Exámenes al finalizar cada semana, es decir, los viernes
		\item Al final del curso haremos una investigación a cerca del tema que nos parezca conveniente o del agrado del alumno, en el cual se aplique lo visto durante las quince clases del curso. Se deberá hacer una presentación a cerca del tema elegido
		
Para el curso el profesor nos recomendó un libro que podemos encontrar en el siguiente enlace: www.blibliotecas.unam.mx 

La clase prosiguió y se nombraron algunas sistemas operativos entre ellos: Windows, Linux (el que se usará para el curso), iOs
También se nombraron distintos lenguajes de programación: pyton, C, C++, java entre otros.
El profesor nos dio información básica acerca de la computación en general como del lenguaje de progración
Por ejemplo las distintas versiones de linux que son: Ubuntu, Fedora,Debian y Slakw.
También nos explicó que un bit es un uno o un cero, ya que así, en sistema binario es como se representa la información dentro de una computadora
 En Linux existen tres tipos de permisos: Usuario (rwx), Grupo (rw-) y Todos (r--)
 También nos dio algunos comandos para ejecutar en la terminal
 \begin{enumerate}
 	\item touch /tmp/algo
 	\item ls - l /tmp/algo
 	\item chmod 751 /tmp/algo
 	\item /tmp/algo
 	
Tanto en windows como en Linux existen las variantes de entorno conociadas como PATH
Algunos comandos para utilizar el path son "set" que sirve para ver la variante de entorno y "pwd" para saber en qué dirección se encuentra. 
Finalizamos la clase hablando un poco acerca de pyton.
 \end{enumerate}
		\end{enumerate}
\end{enumerate}	
			


\end{document}
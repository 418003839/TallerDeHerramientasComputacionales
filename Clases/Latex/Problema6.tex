\documentclass[letterpaper, 12pt, oneside]{article}%para dar formato al documento
\usepackage{amsmath}
\usepackage{graphicx}
\usepackage{xcolor}
\usepackage[utf8]{inputenc}
\usepackage{enumitem}

%Aquí inicia la portada de mi documentos
\title{\Huge Taller de Herramientas Computacionaes}
\author{Jorge S. Martínez Villafan}
\date{Enero 21, 2019}

\begin{document}
\maketitle
%includegraphics[scale=0.4]{1.png}
\newpage
\title{Resumen del problema 2}
\section{Tarea4}
\textbf En el problema 6 lo que hice fue perdir al usuario 10 datos, cada dato iba almacenado en una asignacion a, b,..,j, se definía la suma como la suma de a+b,..,+j y al resultado de esta suma se dividía entre 10, a esta función se le llamó promedio 
\section{Tarea5}
Hice una lista vacía, después puse input para que el usuario metiera los datos que deseaba y la lista vacia fuera llenandose.Puse el ciclo for in para que fuera recorriendo cada uno de los datos, definí una variable para que se fueran almacenado la suma, e hice para sacar el promedio que con float partido por len (la longitud de la lista)nos devolvieran valores con puntos decimales. Tampoco copila
\end{document}
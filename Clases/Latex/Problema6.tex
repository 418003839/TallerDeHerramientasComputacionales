\documentclass[letterpaper, 12pt, oneside]{article}%para dar formato al documento
\usepackage{amsmath}
\usepackage{graphicx}
\usepackage{xcolor}
\usepackage[utf8]{inputenc}
\usepackage{enumitem}

%Aquí inicia la portada de mi documentos
\title{\Huge Taller de Herramientas Computacionaes}
\author{Jorge S. Martínez Villafan}
\date{Enero 21, 2019}

\begin{document}
\maketitle
%includegraphics[scale=0.4]{1.png}
\newpage
\title{Resumen del problema 2}
\section{Tarea4}
\textbf En el problema 6 lo que hice fue perdir al usuario 10 datos, cada dato iba almacenado en una asignacion a, b,..,j, se definía la suma como la suma de a+b,..,+j y al resultado de esta suma se dividía entre 10, a esta función se le llamó promedio 
\section{Tarea5}
Creé una lista con diez daros\\
Definí \color{red} suma \color{black} para que se inicialice en 0, y con ayuda de \color{blue} for in \color{black} para que \color{blue} dato \color{black} recorrienta la lista\\
Definí la suma igual dato más uno y un promedio que era la suma divido por \color{blue} len \color{black} es decir el tamaño de la lista\\
Imprimí la lista 
\end{document}
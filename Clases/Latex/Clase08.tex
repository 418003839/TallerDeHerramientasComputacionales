\documentclass[letterpaper, 12pt, oneside]{article}%para dar formato al documento
\usepackage{amsmath}
\usepackage{graphicx}
\usepackage{xcolor}
\usepackage[utf8]{inputenc}
\usepackage{enumitem}


%Aquí inicia la portada de mi documentos
\title{\Huge Taller de Herramientas Computacionaes}
\author{Jorge S. Martínez Villafan}
\date{Enero 11, 2019}

\begin{document}
\maketitle
%includegraphics[scale=0.4]{1.png}
\newpage
\title{Clase numero ocho}

\textbf En la clase del miercoles, es decir la clase numero siete del curso iniciamos conocientos varios comandos de la terminal que nos sirven para abrir pyhton en segundo plano, esto nos permite seguir escribiendo en la terminal donde ejecutamos el idle para abrir python
ALgunos comandos que vimos fueron los siguientes:
\begin{enumerate}
\item ctrl c: termina el programa
\item kill -9 mata el programa que se esta ejecutando
\item chmod +x "nombre de un archivo" para darle los permisos, es decir, leer, escribir, ejecutar etc
\item find . -name "*.es" nos muestra todos los archivos .py que tengamos en la carpeta en que estemos ubicados
\item where is python nos muestra el lugar donde se encuentra pyhton
\end{enumerate}
Proseguimos la clase hablando de objetos y metodos con las siguientes definiciones:\\
Un objeto es un conjunto de datos y funciones relacionadas que posee caracteristicas. Un objeto puede ser un árbol, como caracteristica tiene hojas\\
Metodo es una acción que realiza el objeto, continuando con el ejemplo del árbol, este realiza la acción de alimentarse mediante el método de la fotosintesis. \\
Clase: una clase es un conjunto para la creación de objetos de datos según un modelo predefinido.

Continuamos la clase trabajando con python, el profesor nos inidico que debemos iniciar los trabajos en python de la siguiente manera:\\

\#!/usr/bin/python2.7\\
\# -*- coding: utf-8 -*-\\
"""\\
Jorge S Martínez Villafan, 418003839\\
Herramientas computacionales\\
Lo que nos explico el miercoles\\
"""\\
Obviamente este es mi caso, los demás deberan llenarlo con sus datos.
Continuamos la clase trabajando con import math sqrt para poder sacar la raíz cuadrada de un numero, en ese caso era la del número 16. Hicimos un ejercicio llamado Ulam, que se trata sobre si es par dividir en dos, si es impar multiplicarl por 3 y añadirle uno. \\
Una vez terminamos de trabjar con Python, nuevamente fue turno de Karla quién nos habló un poco sobre TeXstudio, pero el tiempo no nos dejó avanzar mucho. La próxima clase la iniciaría ella.
\end{document}
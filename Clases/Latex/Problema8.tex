\documentclass[letterpaper, 12pt, oneside]{article}%para dar formato al documento
\usepackage{amsmath}
\usepackage{graphicx}
\usepackage{xcolor}
\usepackage[utf8]{inputenc}
\usepackage{enumitem}

%Aquí inicia la portada de mi documentos
\title{\Huge Taller de Herramientas Computacionaes}
\author{Jorge S. Martínez Villafan}
\date{Enero 21, 2019}

\begin{document}
\maketitle
%includegraphics[scale=0.4]{1.png}
\newpage
\title{Resumen del problema 8}
\section{Tarea4}
\textbf En el problema 8 definí una clave y una i, con esto hice un clico while en el que la i debía ser menor a 3, si (if) la clave era distinta de una contraseña que pedía con un input se imprimiría el mensaje contraseña incorrecta. Sino (else), contraseña correcta.
Si se rebazaran los 3 intentos se imprimiría un mensaje "Se excedieron el numero de intentos"

\section{Tarea5}
En el problema 8 definí una clave y una i, con esto hice un clico \color{blue} while \color{black} en el que la i debía ser menor a 3 Creé una lista vacia y condicioné si \color{blue} if \color{black} la clave era distinta de una contraseña que pedía con un input se imprimiría el mensaje contraseña incorrecta. Pero la ultima contraseña que el usuario metiera se guardaria en una asignación. Sino \color{blue} else \color{black}, contraseña correcta. También la contraseña se almacenaria en una asignación \color{red} a \color{black}
Si se rebazaran los 3 intentos se imprimiría un mensaje "Se excedieron el numero de intentos. Pedí que se imprimera la lista con la ultima contraseña
\end{document}
\documentclass[letterpaper, 12pt, oneside]{article}%para dar formato al documento
\usepackage{amsmath}
\usepackage{graphicx}
\usepackage{xcolor}
\usepackage[utf8]{inputenc}
\usepackage{enumitem}

%Aquí inicia la portada de mi documentos
\title{\Huge Taller de Herramientas Computacionaes}
\author{Jorge S. Martínez Villafan}
\date{Enero 17, 2019}

\begin{document}
\maketitle
%includegraphics[scale=0.4]{1.png}
\newpage
\title{Clase número nueve}

\textbf La clase número nueve, por primera vez fue iniciada por la ayudante Karla. Iniciamos aprendiendo a poner tablas en TeXstudio, hicimos una tabla en la cual iba seccionada con algunos carácteres como las barras del valor absoluto encerrado a una C, para crear una columna, o escribiendo hline para poner una línea horizontal
Continuamos descubriendo nuevos paquetes del TeXstudio, por ejemplo:
\begin{enumerate}
\item documentclass{book}, sirve para darle formato de libro al archivo que estamos haciendo
\item usepackage[spanish]{babel}, sirve para que nos lanze en español las secciones que el libro crea
\item usepackage{biblatex} para crear la bibliografia
\item usepackage{hyperref} nos permite poder hacer un enlace a un link deseado
\end{enumerate}
Algunos otros comandos que conocimos de Latex en aquella clase fueron: 
\begin{enumerate}
	\item section*{Introdución} con el uso de astedisco hace aparecer un número de conteo
	\item url, sirve para poder crear un enlace a un link, similar al "hyperref"
	\item chapter, sirve para poder crear nuevos caítulos dentro del libro
	\item begin{verbatim}, este tiene un uso muy importante, nos permite leer un código de python (desconozco si funciona para otros lenguajes de programación), tal y como fue escrito, sin que TeXstudio nos marque error por carácteres propios del programa
	\item input, sirve para poder abrir un documento dentro de TeXstudio
\end{enumerate}
Una vez terminamos la parte de LateX, fue turno del profesor, quién nos preguntó cómo nos fue con la tarea de los diez problemas que dejó la clase pasada, a la mayoría le fue mal, así que el profesor empezó a resolver el primer problema, el del máximo común divisor por el método de euclides.\\
Primero planteó el problema en el pizarrón y con ayuda del grupo fue dandole forma a lo que sería el código. \\
El código tan solo utilizaba una definición, un while y un return.\\
En el archivo que lleva la S, nos enseño que solo bastaba con importar la def utilizada en el código, y con ayuda de input podíamos hacer interactivo el programa\\

Nos pidió terminar los diez problemas que habían quedado de tarea.
La clase finalizó
\end{document}
\documentclass[letterpaper, 12pt, oneside]{article}%para dar formato al documento
\usepackage{amsmath}
\usepackage{graphicx}
\usepackage{xcolor}
\usepackage[utf8]{inputenc}
\usepackage{enumitem}

%Aquí inicia la portada de mi documentos
\title{\Huge Taller de Herramientas Computacionaes}
\author{Jorge S. Martínez Villafan}
\date{Enero 21, 2019}

\begin{document}
	\maketitle
	%includegraphics[scale=0.4]{1.png}
	\newpage
	\title{Resumen del problema 4}
	\section{Tarea4}
	\textbf En el problema numero 4 definí una función a la cual llamé "fibo", además definí los primeros valores de la seria, a y b, con 0 y 1 respectivamente"\\
	Creé una ciclo whie con a menor a n,  definí un print, y asigné que los valores de a y b cambiaran porque así cambían en la serie, el a se convierte en el b anterior y el b en a mas b. \\
	puse la función y una n que quisiera y lo imprimí
	
	\section{Tarea5}
	Depués de definir una lista vacia, di los valores inicias \color{red} a,b \color{black} e hice una i para que empezara a correr desde 0 pero siempre menor que n, hice que se cambiaran los valores, a tomara el de b y así. Con ayuda del comando \color{blue} ".extend" \color{black} hice que los valores que se fueran generando se fueran añadiendo a la lista vacia que en un principio creé. 
\end{document}
%\documentclass{article}
\documentclass[letterpaper, 12pt, oneside]{article}%para dar formato al documento
\usepackage{amsmath}
\usepackage{graphicx}
\usepackage{xcolor}
\usepackage[utf8]{inputenc}
\usepackage{enumitem}


%Aquí inicia la portada de mi documentos
\title{\Huge Taller de Herramientas Computacionaes}
\author{Jorge S. Martínez Villafan}
\date{Enero 19, 2019}

\begin{document}
\maketitle
%includegraphics[scale=0.4]{1.png}
\newpage
\title{Cuestionario de la semada dos}

\textbf{Segundo cuestionario semanal}
\begin{enumerate}
	\item ¿Puedes siempre obtener un valor exacto cuando ocupas la computadora?\\
R: No, pero puedes obtener una aproximación que satisfaga lo que necesitas
	\item ¿Para qué sirve el comando if?\\ R: sirve para dar una sentencia, ejemplo: "Si sucede tal cosa" entonces  pasa esto
	\item ¿Y el comando else?\\
R: Es la condicionante del if, si no sucede lo del if, sucede lo del else
	\item ¿Se puede usar un un if sin un else y viceversa?\\
R: Sí se puede usar un if sin un else, pero no un else sin un if
	\item ¿Para qué sirve el comando return?\\
R: Nos regresa un valor que previamente se asignó
	\item ¿Para qué sirve el comando while?\\
R: Crea un ciclo
	\item ¿En python son importante el uso de bloques y espacios?\\
R: Sí, cada bloque define algo que queremos que suceda y están delimitados por 4 espacios de diferencia, por lo que sí importan los espacios. Salvo en las asignaciones pero se recomienda su uso

\item ¿Para qué sirven los \$ \$ en TeXstudio?\\
R: sirven para indicar que escribiremos expresiones, matemáticas
\item ¿Cómo ponemos subindices?\\
R: \$x\_{a}\$ para poner subindices, en este caso, es equis subindice a
\item ¿Cómo expresamos una potencia?\\ \$x\^{2}\$ sirven para poner exponentes, en este caso equis al cuadrado
\item ¿Como ponemos una facción en LateX?\\ R: \$\\frac{a}{b}\$ sirven para poner fraciones.
\item ¿ Para que sirve begin{bmatrix}?\\ R: sirve para poner matrices
\item ¿Cómo deben diferenciarse los terminos en las matrices?\\
R: los terminos de éstas deben ir con un \& entre ellos para poder diferenciarlos
\item ¿Para qué sirve dots?\\
R: Pone puntos suspensivos
\item ¿y vdots?\\
R: pone puntos suspensivos  verticales
\item ¿Cómo escirbimos una raíz cuadrada?\\ R: con sqrt para poder escribir raices (similar a pyhon)
\item ¿Cómo escribimos una integral en LateX? \\
R: \$\\int\_{a}\^{b}x\^2 dx, para poder escribir integrales, en este caso, la integral de a a b, de equis cuadrada
\item ¿Dentro de la terminal que sucede si presionas ctrl más c?\\
R: termina el programa
\item ¿para qué sirve el comando kill -9?\\
R: mata el programa que se esta ejecutando
\item ¿Y el comando chmod +x "nombre de un archivo"?\\
R: da permisos, es decir, leer, escribir, ejecutar etc
\item ¿Para qué sirve el comando find . -name "*.es"?\\
R: nos muestra todos los archivos .py que tengamos en la carpeta en que estemos ubicados
\item ¿Cómo vemos el lugar donde se encuentra pyhton? \\ 
R:Con el comando where is python

	\item Da un ejemplo de un objeto\\
R: Computadora
	\item Da una caracteristica del objeto anterior:\\
R: Tiene una memoria Ram
	\item Da un método que realice el objeto anteriormente mencionado \\
	\item ¿De qué sirve poner \#!/usr/bin/python2.7 al inicio de un archivo en python?\\
R: Sirve para que el archivo pueda ser leído y ejecurado en la terminal
\item ¿Para qué sirve el paquete documentclass{book} en TeXstudio?\\
R: sirve para darle formato de libro al archivo que estamos haciendo
\item ¿Y el paquete usepackage[spanish]{babel}?\\
R: sirve para que nos lanze en español las secciones que el libro crea
\item ¿Para que se usa el paquete usepackage{biblatex}?\\
R: para crear la bibliografia
\item ¿Y el paquete usepackage{hyperref}?\\
R: nos permite poder hacer un enlace a un link deseado
	\item ¿Qué sucede si en "section*{Introdución}" se pone un astedisco?\\
R: con el uso de astedisco hace aparecer un número de conteo
\item  ¿Cómo ponemos un elenace dentro de TeXstudio?\\
R: con \ URl
\item ¿Como creamos un nuevo capitulo dentro del libro?\\ 
R: Con chapter
\item ¿Para qué sirve begin{verbatim}?\\
R: nos permite leer un código de python tal y como fue escrito, sin que TeXstudio nos marque error por carácteres propios del programa
\item ¿Para que sirve += dentro de python?\\
R: sirve para hacer una suma y arroje el resultado
\item ¿y !=?\\
R: sirve como "diferente de"
\item ¿Y *=?\\
R: sirve para hacer una multiplicación y arroje el resultado
\item ¿Y /=?\\
R: sirve para hacer una divisisón y arroje el resultado
\item ¿Para que se usa bool?
R: sirve para saber que contiene una cadena, da falso si es vacia, da true si tiene algo. También se puede hacer una asignación previa, bool de la asignación dará truesi si el valor de la asignación es distinto de 0, de lo contrario dará false
\item ¿Para que sirve L.append?
R: para agregar elementos a un lista, es del tipo objeto metodo
\item ¿Para que se usa L.append /([])?\\
R: para meter una lista dentro de otra
\item ¿Como sabemos el elemento del indice x?\\
R: Con L[x]
\item ¿Para qué sirve len(l)?\\
R: para saber cuantos elementos tiene una lista
\item ¿Qué hacemos con L[X].append?\\
R: Se puede agregar más valores a la lista dentro de otra lista
\item ¿Y con L.insert?\\
R: Sirve para agregar algo antes del indice que queramos, el indice es la posición de un elemento de la lista
\item ¿Para qué sirve r=L.pop()?\\
R: se saca el ultimo elemento de la lista
\item ¿Para qué sirve r=L.pop(x)?\\
R: elimina el elemento del indice x
\item ¿Para qué sirve L.extend?\\
R: es para agregar varios elementos a la lista

	\item Aparte de while, ¿Existe otro comando para crear ciclos en python?\\
R: sí, con el comando for in podemos crear ciclos
	\item ¿Para qué sirve range(x)?\\
R: Crea intervalos acotados en los valores que le indiques
	
\end{enumerate}
\end{document}